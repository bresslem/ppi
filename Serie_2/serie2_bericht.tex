\documentclass{scrartcl}
\usepackage[utf8]{inputenc}
\usepackage[T1]{fontenc}
\usepackage[ngerman]{babel}
\usepackage{amssymb}
\usepackage{amsmath}
\usepackage{graphicx}
\usepackage{framed}
\usepackage{xcolor}
\usepackage[nottoc]{tocbibind}
\usepackage{caption}
\usepackage{setspace}
\onehalfspacing

\colorlet{shadecolor}{gray!25}
\setlength{\parindent}{0pt}

\newcommand{\abs}[1]{\left\lvert#1\right\rvert}
\newcommand{\R}{\mathbb{R}}

\begin{document}

\title{Darstellung von Differentialgleichungen als lineare Gleichungssysteme}
\author{Marisa Breßler und Anne Jeschke}
\date{29.11.2019}
\maketitle

\tableofcontents

\pagebreak
\section{Einleitung}

\pagebreak
\section{Theorie}
Im Allgemeinen betrachtet man bei der Lösung des Poisson-Problems ein Gebiet $\Omega\subset\R^d$ und dessen Rand $\partial\Omega$.
Auf diesem Gebiet sind die zwei Funktionen $f\in C(\Omega ; \R)$ und $g\in C(\partial\Omega ; \R)$ gegeben.
Das Poisson-Problem beschreibt die Suche nach der Lösung $u$ einer elliptischen partiellen Differentialgleichung (PDE) der Form
\[\begin{split}
-\Delta u &= f \textrm{ in } \Omega\\
        u &= g \textrm{ in } \partial\Omega
\end{split}
\]
Hierbei beschreibt $\Delta u$ den Laplace-Operator, der für eine Funktion $u\in C^2(\R^d;\R)$ definiert ist durch \[\Delta u := \sum_{l=1}^{d} \frac{\partial^2 u}{\partial x^2_l}\]
Das heißt für $d=1$ gilt $\Delta u = u''$ und für $d>1$ ist der Laplace-Operator die Summe aller partiellen Ableitungen von $u$ zweiter Ordnung, die zweimal nach der selben Variable ableiten.

Im Rahmen dieses Praktikums werden wir uns konzentrieren auf den Fall $\Omega=(0,1)^d$, $g\equiv0$ und $d\in\{1, 2, 3\}$, wobei sich unsere Ergebnisse auch verallgemeinern lassen auf $d>3$.

Um das Problem mit den Mitteln der Numerik lösen zu können, diskretisieren wir sowohl das Gebiet $\Omega$, durch eine Auswahl von endlich vielen Punkten $x_1$ bis $x_N$, an denen wir die Funktion approximieren wollen, als auch den Laplace-Operator, mithilfe von finiten Differenzen.
Dadurch ergibt sich dann ein lineares Gleichungssystem $A^{(d)}\hat{u}=b$. Die linke Seite entspricht dann dem diskretisierten Laplace-Operator, wobei der Vektor $\hat{u}\in\R^N$ die gesuchte Funktion $u$ an den $N$ ausgewählten Diskretisierungspunkten approximiert.

\subsection{Diskretisierung des Gebietes}
Da wir nur endlich viele Punkte auswerten können, müssen wir zuerst das Gebiet, das wir untersuchen, diskretisieren und nur bestimmte Stellen bestimmen, mit deren Hilfe wir die Lösung des Problems approximieren.
Im eindimensionalen Fall teilen wir das Intervall $\Omega=(0,1)$ dafür äquidistant in $n$ Teilintervalle der Länge $h:=\frac{1}{n}$ und erhalten somit die $N=n-1$ Diskretisierungspunkte in
$X_1:=\{\frac{j}{n} | 1\leq j \leq n-1\}$.

Im mehrdimensionalen Fall $d>1$ untersuchen wir analog die Punkte des $d$-fachen kartesischen Produktes dieser Menge $X_d := X_1^d$. Im zweidimensionalen Fall also das uniforme Gitter $X_2 := X_1^2 = \{(\frac{j}{n}, \frac{k}{n}) | 1\leq j,k \leq n-1\}$.

Insgesamt erhalten wir in jeder Dimension $(n-1)^d$ Diskretisierungspunkte.

\subsection{Diskretisierung des Laplace-Operators}
Wir wollen den Laplace-Operator für $u$ mithilfe der schon bekannten zweiten finiten Differenz diskretisieren. Im eindimensionalen Fall lautet diese für einen Diskretisierungspunkt $x_i\in X_1$ mit $i \in \{1,...,N\}$:
$D_h^{(2)}u(x_i) = \frac{u(x_i+h) - 2u(x_i)+u(x_i-h)}{h^2}$ und entspricht dem diskreten Laplace-Operator für $u(x_i)$, den wir hier mit $\Delta_h u(x_i)$ bezeichnen.

Im mehrdimensionalen Raum, das heißt für $d>1$ entspricht der diskrete Laplace-Operator der Summe der zweiten finiten Differenzen in alle Richtungen.

Sei $e_l$ für $1\leq l \leq d$ der $l$-te Einheitsvektor in $\R^d$. Definiert man die Abbildung $u_{x,l}:\R\to\R$ als $u_{x,l}(t):=u(x+te_l)$, dann kann man die zweite finite Differenz von $u$ an der Stelle $x$ in Richtung $e_l$ schreiben als:
\[\frac{\partial^2 u(x)}{\partial x_l^2}\approx D_h^{(2)}u_{x,l}(0)\]

Den diskreten Laplace-Operator, die Summe der zweiten finiten Differenzen in alle Richtungen, definieren wir dementsprechend wie folgt:
\[\Delta_h u(x):= \sum_{l=1}^{d} D_h^{(2)}u_{x,l}(0)\]

Nach der Diskretisierung des Gebietes gilt $h=\frac{1}{n}$ und damit gilt für alle $x\in X_d$, dass $x\pm he_l$ entweder einer der benachbarten Diskretisierungspunkte ist, also $x\pm he_l\in X_d$, oder auf dem Rand des Gebietes $\partial\Omega$ liegt und dementsprechend gilt $u(x\pm he_l)=0$.
Wir approximieren also den Laplace-Operator von $u$ an der Stelle $x$ mithilfe der Werte von $u$ an den zu $x$ benachbarten Diskretisierungspunkten.

Für $d=2$ und $x:=(x_1,x_2)\in X_2$ gilt:
\[\Delta_h u(x) = \frac{4u(x_1,x_2)-u(x_1-he_1,x_2)-u(x_1+he_1,x_2)-u(x_1,x_2-he_2)-u(x_1,x_2+he_2)}{h^2}\]

Für $d=3$ steht im Zähler entsprechend der sechsfache Wert von $u(x)$ und man subtrahiert die Werte der beiden benachbarten Punkte in allen drei Dimensionen.

\subsection{Aufstellen eines Linearen Gleichungssystems}
Da wir $N=(n-1)^d$ verschiedene Diskretisierungspunkte gewählt haben erhalten wir ein Gleichungssystem mit $N$ Gleichungen und der Form $A^{(d)}\hat{u}=b$, wobei der Vektor $\hat{u}\in\R^n$ die Werte der Funktion $u$ an den $N$ Diskretisierungspunkten approximieren soll.

Den diskreten Laplace-Operator stellen wir dar, indem wir den Vektor $\hat{u}$ mit der Matrix $A^{(d)}$ multiplizieren.

Für $d=1$ gilt:
\[A^{(d)}=
\begin{pmatrix}
   2 & -1 &  0 & \cdots & 0 \\
  -1 &  2 & -1 & \cdots & 0 \\
   \vdots & \ddots & \ddots & \ddots & \vdots \\
   0 & \cdots & -1 &  2 & -1 \\
   0 & 0 & \cdots & -1 &  2
\end{pmatrix}
\in\R^{N\times N}
\]
und dementsprechend enthält
\[A^{(d)}\hat{u}=
\begin{pmatrix}
   2 & -1 &  0 & \cdots & 0 \\
  -1 &  2 & -1 & \cdots & 0 \\
   \vdots & \ddots & \ddots & \ddots & \vdots \\
   0 & \cdots & -1 &  2 & -1 \\
   0 & 0 & \cdots & -1 &  2
\end{pmatrix}
\begin{pmatrix}
  u(x_1)\\
  u(x_2)\\
  \vdots\\
  u(x_{N-1})\\
  u(x_N)
\end{pmatrix}
=
\begin{pmatrix}
  2u(x_1)-u(x_2)\\
  -u(x_1)+2u(x_2)-u(x_3)\\
  \vdots\\
  -u(x_{N-2})+2u(x_{N-1})-u(x_N)\\
  2u(x_{N-1})-u(x_N)
\end{pmatrix}
\]
die zweiten finiten Differenzen für alle $x\in X_1$.

Für $d > 1$ mu
\pagebreak
\section{Experimente und Beobachtungen}


\pagebreak
\section{Auswertung}

\pagebreak
\section{Zusammenfassung und Ausblick}

\pagebreak
\bibliographystyle{plain}
\bibliography{serie2_literatur}

\end{document}