\documentclass{scrartcl}
\usepackage[utf8]{inputenc}
\usepackage[T1]{fontenc}
\usepackage[ngerman]{babel}
\usepackage{amssymb}
\usepackage{amsmath}
\usepackage{graphicx}
\usepackage{framed}
\usepackage{xcolor}
\usepackage[nottoc]{tocbibind}
\usepackage{caption}

\colorlet{shadecolor}{gray!25}
\setlength{\parindent}{0pt}

\newcommand{\abs}[1]{\left\lvert#1\right\rvert}
\newcommand{\R}{\mathbb{R}}

\begin{document}

\title{Darstellung von Differentialgleichungen als lineare Gleichungssysteme}
\author{Marisa Breßler und Anne Jeschke}
\date{29.11.2019}
\maketitle

\tableofcontents

\pagebreak \section{Einleitung}
\label{sec:einleitung}

\pagebreak \section{Theorie}
\label{sec:theorie}
Im Allgemeinen betrachtet man bei der Lösung des Poisson-Problems ein Gebiet $\Omega\subset\R^d$ und dessen Rand $\partial\Omega$.
Auf diesem Gebiet sind die zwei Funktionen $f\in C(\Omega ; \R)$ und $g\in C(\partial\Omega ; \R)$ gegeben.
Das Poisson-Problem beschreibt die Suche nach der Lösung $u$ einer elliptischen partiellen Differentialgleichung (PDE) der Form
\[\begin{split}
-\Delta u &= f \textrm{ in } \Omega\\
        u &= g \textrm{ in } \partial\Omega
\end{split}
\]
Hierbei beschreibt $\Delta u$ den Laplace-Operator, der für eine Funktion $u\in C^2(\R^d;\R)$ definiert ist durch \[\Delta u := \sum_{l=1}^{d} \frac{\partial^2 u}{\partial x^2_l}\]
Das heißt für $d=1$ gilt $\Delta u = u''$ und für $d>1$ ist der Laplace-Operator die Summe aller partiellen Ableitungen von $u$ zweiter Ordnung, die zweimal nach der selben Variable ableiten.

Im Rahmen dieses Praktikums werden wir uns konzentrieren auf den Fall $\Omega=(0,1)^d$, $g\equiv0$ und $d\in\{1, 2, 3\}$, wobei sich unsere Ergebnisse auch verallgemeinern lassen auf $d>3$.

Um das Problem mit den Mitteln der Numerik lösen zu können, diskretisieren wir sowohl das Gebiet $\Omega$, als auch den Laplace-Operator.
\pagebreak \section{Experimente und Beobachtungen}
\label{sec:experimente}


\pagebreak \section{Auswertung}
\label{sec:auswertung}

\pagebreak \section{Zusammenfassung und Ausblick}
\label{sec:zusammenfassung}

\pagebreak
\bibliographystyle{plain}
\bibliography{serie2_literatur}

\end{document}