\documentclass{scrartcl}
\usepackage[utf8]{inputenc}
\usepackage[T1]{fontenc}
\usepackage[ngerman]{babel}
\usepackage{amssymb}
\usepackage{amsmath}
\usepackage{algorithmicx}
\usepackage{algpseudocode}
\usepackage{graphicx}
\usepackage{framed}
\usepackage{xcolor}
\usepackage[nottoc]{tocbibind}
\usepackage{caption}
\usepackage{setspace}
\onehalfspacing

\colorlet{shadecolor}{gray!25}
\setlength{\parindent}{0pt}

\newcommand{\abs}[1]{\left\lvert#1\right\rvert}
\newcommand{\R}{\mathbb{R}}

\begin{document}

\title{Lösen des Poisson-Problems mittels Finite-Differenzen-Diskretisierung und LU-Zerlegung}
\author{Marisa Breßler und Anne Jeschke (PPI27)}
\date{03.01.2020}
\maketitle

\tableofcontents

\pagebreak
\section{Einleitung}
Im vorherigen Bericht haben wir bereits das Verfahren vorgestellt, mit dem wir mittels Finite-Differenzen-Diskretisierung das Poisson-Problem lösen.
Dazu haben wir ein lineares Gleichungssystem aufgestellt, dessen Lösung eine numerische Approximation der exakten analytischen Lösung des Poisson-Problems darstellt.
In diesem Bericht möchten wir nun ein Verfahren vorstellen, mit dem wir dieses Gleichungssystem lösen können und die Ergebnisse dieses Verfahrens weiter untersuchen.

\pagebreak
\section{Theorie}

\subsection{LU-Zerlegung}


\pagebreak
\section{Experimente und Beobachtungen}

\pagebreak
\section{Auswertung}

\pagebreak
\section{Zusammenfassung und Ausblick}


\pagebreak

\bibliographystyle{plain}
\bibliography{serie3_literatur}

\end{document}