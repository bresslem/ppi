\documentclass{scrartcl}
\usepackage[utf8]{inputenc}
\usepackage[T1]{fontenc}
\usepackage[ngerman]{babel}
\usepackage{amssymb}
\usepackage{amsmath}
\usepackage{algorithmicx}
\usepackage{algpseudocode}
\usepackage{graphicx}
\usepackage{framed}
\usepackage{xcolor}
\usepackage[nottoc]{tocbibind}
\usepackage{caption}
\usepackage{setspace}
\onehalfspacing

\colorlet{shadecolor}{gray!25}
\setlength{\parindent}{0pt}

\newcommand{\abs}[1]{\left\lvert#1\right\rvert}
\newcommand{\R}{\mathbb{R}}

\begin{document}

\title{Lösen des Poisson-Problems mittels Finite-Differenzen-Diskretisierung\\
und LU-Zerlegung}
\author{Marisa Breßler und Anne Jeschke (PPI27)}
\date{03.01.2020}
\maketitle

\tableofcontents

\pagebreak
\section{Einleitende Worte}
In unserem Bericht  vom 29.11.2019 haben wir das Poisson-Problem vorgestellt und einen numerischen Lösungsansatz aufgezeigt, der es mittels einer Diskretisierung des Gebietes und des Laplace-Operators in das Lösen eines linearen Gleichungssystems überführt.
Letzteres soll nun wie angekündigt durchgeführt werden.
In dieser Arbeit wollen wir das lineare Gleichungssystem direkt lösen.
Dazu nutzen wir die LU-Zerlegung (mit Spalten- und Zeilenpivotisierung) der ermittelten tridiagonalen Block-Matrix $A^d$.

Anhand einer Beispielfunktion und den bereits im vorherigen Bericht betrachteten Fällen des Einheitsintervalls, -quadrates, -würfels (d.h. für das Gebiet ... gilt:
...)
wollen wir im Folgenden die Funktionalität (Genauigkeit/Fehler, Konvergenzgeschwindigkeit, Effizienz) dieses Lösungsverfahrens exemplarisch untersuchen.
Alle im Rahmen dessen nötigen theoretischen Grundlagen finden sich in unseren vorherigen Berichten.

\pagebreak
\section{Untersuchungen zur Genauigkeit}
Für unsere Untersuchungen wählen wir folgende Beispielfunktion:
...

Gesamtfehler = Verfahrens-/Approximationsfehler + Rundungsfehler

\subsection{Verfahrens-/Approximationsfehler}
Genauigkeit der numerischen Lösung umso höher, je kleiner Intervalllänge $h=1/n$ bzw. je größer Anzahl der Intervalle (in jeder Dimension) oder der Diskretisierungspunkte

beispielhaft für Fall $d=2$

grafische Darstellung der Lösung

selbst bei sehr grober Diskretisierung ($n=5$) Abweichung/Fehler mit Auge kaum wahrzunehmen > deswegen Differenz auch im Plot

Zusammenhang: je größer $N$, desto genauer Lösung/kleiner Fehler

\subsection{Rundungsfehler}
Lösen eines linearen Gleichungssystems beschreibt mathematisches Problem

Kondition einer Matrix: Maß für Abhängigkeit der Lösung eines Problems von Störung der Eingangsdaten

Konditionszahl: Faktor, um den sich der Eingangsfehler maximal verstärken kann 

\pagebreak
\section{Untersuchungen zum Speicherplatz}


\pagebreak
\section{Zusammenfassung und Ausblick}


\pagebreak

\bibliographystyle{plain}
\bibliography{serie3_literatur}

\end{document}