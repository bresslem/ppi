\documentclass{scrartcl}
\usepackage[utf8]{inputenc}
\usepackage[T1]{fontenc}
\usepackage[ngerman]{babel}
\usepackage{amssymb}
\usepackage{amsmath}
\usepackage{algorithmicx}
\usepackage{algpseudocode}
\usepackage{graphicx}
\usepackage{framed}
\usepackage{xcolor}
\usepackage[nottoc]{tocbibind}
\usepackage{caption}
\usepackage{setspace}
\onehalfspacing

\colorlet{shadecolor}{gray!25}
\setlength{\parindent}{0pt}

\newcommand{\abs}[1]{\left\lvert#1\right\rvert}
\newcommand{\R}{\mathbb{R}}

\begin{document}

\title{Lösen des Poisson-Problems mittels Finite-Differenzen-Diskretisierung\\
und LU-Zerlegung}
\author{Marisa Breßler und Anne Jeschke (PPI27)}
\date{03.01.2020}
\maketitle

\tableofcontents

\pagebreak
\section{Einleitende Worte}
In unserem Bericht  vom 29.11.2019 haben wir das Poisson-Problem vorgestellt und einen numerischen Lösungsansatz aufgezeigt, der es mittels einer Diskretisierung des Gebietes und des Laplace-Operators in das Lösen eines linearen Gleichungssystems überführt.
Letzteres soll nun wie angekündigt durchgeführt werden.
In dieser Arbeit wollen wir das lineare Gleichungssystem direkt lösen.
Dazu nutzen wir die LU-Zerlegung (mit Spalten- und Zeilenpivotisierung) der ermittelten tridiagonalen Block-Matrix $A^d$.

Anhand einer Beispielfunktion und den bereits im vorherigen Bericht betrachteten Fällen des Einheitsintervalls, -quadrates, -würfels (d.h. für das Gebiet $\Omega\subset\R^d$ ($d\in\mathbb{N}$) und dessen Rand $\partial\Omega$ gilt:
$\Omega=(0,1)^d$, $d\in\{1, 2, 3\}$ mit der Randbedingung $u \equiv 0$ auf $\partial\Omega$, wobei $u$ die gesuchte Funktion ist)
wollen wir im Folgenden die Funktionalität (Genauigkeit/Fehler, Konvergenzgeschwindigkeit, Effizienz) dieses Lösungsverfahrens exemplarisch untersuchen.
Alle im Rahmen dessen nötigen theoretischen Grundlagen finden sich in unseren vorherigen Berichten.

\pagebreak
\section{Untersuchungen zur Genauigkeit}
Für unsere Untersuchungen wählen wir die Beispielfunktion $u: \Omega \rightarrow \mathbb{R}$, die wie folgt definiert ist:
\[u(x) := \prod \limits_{l=1}^{d} x_l \, sin(\pi x_l)\]
Dabei sei wie bereits erwähnt $\Omega = (0,1)^d$ und $d\in\{1, 2, 3\}$.
Die Funktion $u$ ist die exakte Lösung des Poisson-Problems, sie wird in der Praxis gesucht. Bekannt ist lediglich die Funktion $f\in C(\Omega ; \R)$ und $\forall \, x \in\Omega$ gelte $-\Delta u(x) = f(x)$. Dementsprechend ist die Funktion $f: \Omega \rightarrow \mathbb{R}$ gegeben durch:
\[f(x) := ... \]

Die Genauigkeit unserer numerischen Lösung des Poisson-Problems -- wir nennen diese gesuchte Funktion $\hat{u}$ (denn sie ist die Approximation der exakten Lösungsfunktion $u$) -- ist abhängig von der Größenordnung der Fehler. Dabei gilt folgender Zusammenhang: Der Gesamtfehler setzt sich aus Verfahrens-/Approximationsfehler auf der einen und Rundungsfehler auf der anderen Seite zusammen. Im Folgenden wollen wir beiden Fehlerarten in Hinblick auf unser Beispiel betrachten.

\subsection{Verfahrens-/Approximationsfehler}
numerisches Verfahren konvergiert: Genauigkeit der numerischen Lösung umso höher, je kleiner Intervalllänge $h=n^{-1}$ bzw. je größer Anzahl der Intervalle (in jeder Dimension) oder der Diskretisierungspunkte

beispielhaft für Fall $d=2$

grafische Darstellung der Lösung (3 Grafiken für n = 5, 10, 20)

selbst bei sehr grober Diskretisierung ($n=5$) Abweichung/Fehler mit Auge kaum wahrzunehmen > deswegen Differenz auch im Plot

Zusammenhang: je größer $N$, desto genauer Lösung/kleiner Fehler

Fehler-/Konvergenzplot (1 Grafik)

Konditionsplot von $A^d$ (1 Grafik)

\subsection{Rundungsfehler}
Lösen eines linearen Gleichungssystems beschreibt mathematisches Problem

Kondition einer Matrix: Maß für Abhängigkeit der Lösung eines Problems von Störung der Eingangsdaten

Konditionszahl: Faktor, um den sich der Eingangsfehler maximal verstärken kann 

Vgl. Kondition von $A^d$ und der entsprechenden Hilbertmatrix mit gleicher Dimension (Tabelle)

\pagebreak
\section{Untersuchungen zum Speicherplatz}
sparsity von $A^d$ und deren LU-Zerlegung (3 Grafiken)

\pagebreak
\section{Zusammenfassung und Ausblick}


\pagebreak

\bibliographystyle{plain}
\bibliography{serie3_literatur}

\end{document}