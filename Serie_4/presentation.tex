\documentclass{beamer}

\usepackage[utf8]{inputenc}

\usetheme{Warsaw}

\title{Hochwasserstände vorhersagen mittels Kleinste-Quadrate-Lösung}
\author{Marisa Breßler und Anne Jeschke}
\date{21.01.2020}

\begin{document}
\maketitle
\frame{\tableofcontents[currentsection]}

\section{Einleitung}
\begin{frame} %%Eine Folie
  \frametitle{Vorstellung des Problems} %%Folientitel
    \begin{table}

    \end{table}
\end{frame}

\section{Theorie und Implementierung}
\begin{frame} %%Eine Folie
  \frametitle{Aufstellung eines Gleichungssystems} %%Folientitel

\end{frame}

\begin{frame} %%Eine Folie
  \frametitle{Methode der kleinsten Quadrate} %%Folientitel

\end{frame}

\begin{frame} %%Eine Folie
  \frametitle{Lösung mittels QR-Zerlegung} %%Folientitel
   Rang von A testen in full\_rank(A) mittels spezieller Form von R
\end{frame}

\section{Experimente}
\subsection{Einfache lineare Regression}

\begin{frame} %%Eine Folie
  \frametitle{Wahl der Experimente} %%Folientitel

\end{frame}

\begin{frame} %%Eine Folie
  \frametitle{Ergebnisse} %%Folientitel

\end{frame}

\begin{frame} %%Eine Folie
  \frametitle{Lineare Mehrfachregression} %%Folientitel

\end{frame}


\end{document}
