\documentclass[smallheadings]{scrartcl}
\usepackage[utf8]{inputenc}
\usepackage[T1]{fontenc}
\usepackage[ngerman]{babel}
\usepackage{amssymb}
\usepackage{amsmath}
\usepackage{hyperref}
\usepackage[pdftex,svgnames,hyperref]{xcolor}
\usepackage{listings}
\definecolor{keywords}{RGB}{255,0,90}
\definecolor{comments}{RGB}{0,0,113}
\definecolor{red}{RGB}{160,0,0}
\definecolor{green}{RGB}{0,150,0}
\lstset{language=Python,
        basicstyle=\ttfamily\small,
        keywordstyle=\color{keywords},
        commentstyle=\color{comments},
        stringstyle=\color{red},
        showstringspaces=false,
        identifierstyle=\color{green},
        }


\usepackage{paralist}

\newcommand{\initem}[2]{\item[\hspace{0.5em} {\normalfont\ttfamily{#1}} {\normalfont\itshape{(#2)}}]}
\newcommand{\outitem}[1]{\item[\hspace{0.5em} \normalfont\itshape{(#1)}]}
\newcommand{\bfpara}[1]{\noindent \textbf{#1:}\,}

\title{Dokumentation zum Modul \texttt{least\_squares.py}}
\author{Marisa Breßler und Anne Jeschke (PPI27)}
\date{\today}

\begin{document}

\maketitle
\tableofcontents

\section{Schnittstellendokumentation computer.py}

\subsection{Methoden}

\subsubsection{\texttt{read\_input(filename, selection=None, number\_of\_columns=3)}}
Diese Methode liest die Eingabe aus einer gegebenen Datei und übersetzt sie in einen mehrsimensionalen Array.
Die Datei enthält für jede Messung eine Zeile und die Messwerte einer Messung sind in der Zeile durch Kommata getrennt.
In unserem Fall sind es immer drei Messwerte pro Messung, dies ist jedoch variabel.

\bfpara{Input}
    \begin{compactdesc}
		    \initem{filename}{String} Name der Eingabedatei
        \initem{selection=None}{list of integers, optional} Liste der Zeilen, die aus der Datei gelesen werden sollen. Falls alle Zeilen gelesen werden sollen, muss dies nicht gesetzt werden.
		\end{compactdesc}
\bfpara{Returns}
    \begin{compactdesc}
		  \outitem{np.ndarray} Eingabedaten als mehrsimensionaler Array
	  \end{compactdesc}

\subsubsection{\texttt{create\_lgs(data, number\_of\_unknowns)}}

\subsubsection{\texttt{create\_lgs\_p2(data)}}

\subsubsection{\texttt{get\_qr(A)}}

\subsubsection{\texttt{full\_rank(A)}}

\subsubsection{\texttt{solve\_qr(A, b)}}

\subsubsection{\texttt{norm\_of\_residuum(A, b)}}

\subsubsection{\texttt{get\_cond(A)}}

\subsubsection{\texttt{get\_cond\_transposed(A)}}

\subsubsection{\texttt{plot\_result(data\_list, labels)}}

\subsubsection{\texttt{plot\_result\_p2(data)}}

\subsubsection{\texttt{plot\_result\_multilinear(data)}}



\end{document}