\documentclass{scrartcl}
\usepackage[utf8]{inputenc}
\usepackage[T1]{fontenc}
\usepackage[ngerman]{babel}
\usepackage{amssymb}
\usepackage{amsmath}
\usepackage{graphicx}
\usepackage{framed}
\usepackage{xcolor}
\colorlet{shadecolor}{gray!25}
\setlength{\parindent}{0pt}

\begin{document}
% ----------------------------------------------------------------------
\title{Approximation von Ableitungen\\mittels finiter Differenzen}
\author{Marisa Breßler und Anne Jeschke}
\date{08.11.2019}
\maketitle
% ----------------------------------------------------------------------------
% Inhaltsverzeichnis:
\tableofcontents
% ----------------------------------------------------------------------------
% Gliederung und Text:
\pagebreak \section{Einleitung}
\label{sec:einleitung}
Im Gegensatz zur Analysis bietet die Numerik praktikable Nährungen an Real- bzw. Idealbilder. Die Güte dieser Näherung ist im Minimum abschätzbar, bleibt also unterhalb einer beweisbaren Toleranzgrenze. \\
Ableitungen von Funktionen spielen in der Praxis eine große Rolle. So dienen die erste und die zweite Ableitung in der Physik zum Beispiel der Untersuchung von Bewegungsabläufen, wo die Geschwindigkeit und Beschleunigung definiert werden als erste und zweite Ableitung des Weges nach der Zeit. Änderungsprozesse müssen in ganz unterschiedlichen Kontexten erfasst und/oder vorausgesagt werden. Dabei ist es in einer Vielzahl von Praxisbeispielen notwendig, Ableitungen zu approximieren. Unter Umständer ist eine Funktion zwar durch Formeln bekannt, doch das exakte Differenzieren gestaltet sich als aufwändig oder das Ermitteln von Funktionswerten als schwierig, weil die Ableitungsfunktion von sehr komplexer Natur ist. Das näherungsweise Differenzieren hat eines ihrer Hauptanwendungsgebiete bei der Verarbeitung von Messwerten. Hier ist die Funktion im Allgemeinen nicht explizit bekannt, das heißt sie liegt nicht in analytischer Form, sondern nur in Form von diskreten Punkten vor. Aufgrund der gegebenen lückenhaften Informationen ist die Ableitung mit analytischen Methoden der Differentialrechnung nicht exakt bestimmbar. An dieser Stelle werden numerische Verfahren verwendet, um die Ableitungen beziehungsweise die Werte der Ableitungen an bestimmten Stellen mit einer gewissen Genauigkeit näherungsweise zu ermitteln. \\
\linebreak
\textit{Numerik beantwortet die Frage: Was bleibt vom Ableitungsbegriff übrig, wenn alle Rechnungen in endlich vielen Schritten und mit endlich vielen Zahlen in endliche vielen Ziffern abgehandelt werden müssen?} (Schneebeli, S. 4)\\
\linebreak
Numerisches Differenzieren ist zum Beispiel mit den sogenannten finiten Differenzen möglich. Diese stellen eine überschaubares Verfahren zum Approximieren von Ableitungen zur Verfügung. Auf welche Weise und wie gut, das heißt mit welcher Genauigkeit, das funktioniert, soll im Folgenden erläutert werden. \\

\pagebreak \section{Theorie}
\label{sec:theorie}
Die Formeln der finiten Differenzen, auch Differenzenformeln genannt, lassen sich auf verschiedene Weisen herleiten. Im Folgenden wollen wir zwei Ansätze vorstellen: Zum einen ist das ein geometrischer Ansatz, der die Definition der Ableitung über den Differentialquotienten nutzt, d.h. den Anstieg der Tangente an der zu untersuchenden Funktion an der zu betrachtenden Stelle. Zum anderen ist es ein Ansatz, der sich die Eigenschaften der Taylorentwicklung zunutze macht.

\subsection{Vom Differenzen- zum Differentialquotienten und umgekehrt}
\label{ssec:herleitung1}
Der Ausgangspunkt unserer geometrischen Herleitung der Differenzenformeln bildet die Definition der Ableitung:
\begin{shaded}
  Eine Funktion $f:D \rightarrow \mathbb{R}$ (mit $D\subset \mathbb{R}$) heißt differenzierbar an der Stelle $x_0 \in D$, falls folgender Grenzwert existiert (mit $(x_0+h) \in D$): \[f'(x_0) = \lim _{x\to x_0} {\frac {f(x)-f(x_0)}{x-x_0}} = \lim _{h\to 0} {\frac {f(x_0+h)-f(x_0)}{h}}\] Dieser Grenzwert heißt \textit{Differentialquotient} / \textit{Ableitung} von $f$ nach $x$ an der Stelle $x_0$.
\end{shaded}
Der Differentialquotient geht zurück auf die Sekantensteigung. Ist die Ableitung einer Funktion $f$ an einer Stelle $x_0$ gesucht, wird wie eingangs erwähnt nach der Steigung der Tangente am Graphen von $f$ im Punkt $(x_0 \mid f(x_0))$ gefragt. Die Tangentensteigung kann näherungsweise mit der Sekantensteigung durch die Punkte $(x_0 \mid f(x_0))$ und $(x_0 + \Delta x \mid f(x_0 + \Delta x))$ bestimmt werden. Die Formel der Sekantensteigung durch die Punkte $(x_0 \mid f(x_0))$ und $(x_0 + \Delta x \mid f(x_0 + \Delta x))$ ist der folgende Differenzenquotient: \[\frac {f(x_0 + \Delta x)-f(\Delta x)}{(x_0 + \Delta x) - x_0} = \frac {f(x_0 + \Delta x)-f(\Delta x)}{\Delta x}\]
Setzt man $\Delta x =: h$ und lässt $h$ gegen $0$ laufen, erhält man die Formel für den Differentialquotienten, sprich für die Ableitung von $f$ an der Stelle $x_0$. Allerdings kann der Computer den Grenzübergang der sogenannten \textit{Schrittweite} $h$ gegen $0$ nicht leisten. Deswegen wählt man den Differenzenquotienten als Näherung der Ableitung. Werden Rundungsfehler vernachlässigt, die ein Computer immer aufgrund seiner begrenzten Genauigkeit in der Zahldarstellung verursacht, geht der numerische Wert des Differenzenquotienten gegen den exakten Wert der Ableitung. Aber auch in der Praxis gilt die Faustregel: Je kleiner die Schrittweite $h$ gewählt wird, desto genauer ist die numerische Näherung. \\
Der Differenzenquotient misst also die mittlere spezifische Änderung von $f$ zwischen zwei Stellen. Man unterscheidet zwischen dem \textit{rechtsseitigem Differenzenquotienten}, bei dem die Steigung der Sekante durch $x$ und $x+h$ berechnet wird, und dem \textit{linksseitigem Differenzenquotienten}, bei dem wiederum die Steigung der Sekante durch $x-h$ und $x$ berechnet wird. Allerdings ist die Nutzung des links- oder des rechtsseitigen Differenzenquotienten, einer \textit{finiten Differenz erster Ordnung} suboptimal, vor allem für einseitig gekrümmte Funktionsgraphen, bei denen oft eine sehr hohe Abweichung zwischen Sekanten- und Tangentensteigung zu beobachten ist. Deshalb kann es unter Umständen sinnvoll sein, diesen Fehler durch Mittelung zu verkleinern und das arithmetische Mittel der beiden einseitigen (asymmetrischen) Differenzenquotienten zu betrachten, das heißt die Sekantensteigung zwischen $x-h$ und $x+h$. Dies bezeichnet man als \textit{zentralen} oder auch \textit{symmetrischen Differenzenquotienten erster Ordnung}. Durch erneutes Anwenden der symmetrischen Differenzenformel für die erste Ableitung lässt sich eine symmetrische Differenzenformel zur Approximation der zweiten Ableitung gewinnen: die \textit{finite Differenz zweiter Ordnung}. Für höhere Ableitungen geht man analog vor. (4 FORMELN) \\

\subsection{Approximieren von Ableitungen via Taylorentwicklung}
\label{ssec:herleitung2}
Auch mithilfe der Taylorentwicklung einer Funktion lassen sich Formeln zur Approximation der Ableitungen dieser Funktion herleiten.\\
Betrachtet man auf einem Intervall $[a,b]\in\mathbb{R}$ eine relle Funktion $f\in C^\infty([a,b])$, so kann man die Ableitung an einer Stelle $x\in(a,b)$ abschätzen, indem man sich zusätzlich eine Stelle $x_+ := x+h\in[a,b]$ mit $0<h\in\mathbb{R}$ hernimmt und den Funktionswert von $f$ an der Stelle $x_+$ mithilfe der Taylorentwicklung
\[f(x_+)=\sum_{n=0}^{\infty}\frac{f^{(n)}(x)}{n!}h^n=f(x)+f'(x)h+\frac{f''(x)}{2}h^2+...\]
approximiert. Wenn man diese entsprechend umstellt erhält man die Formel für die \textit{erste finite Differenz}
\[(D_h^{(1)}f)(x)=\frac{f(x_+)-f(x)}{h}=f'(x)+\sum_{n=2}^{\infty}\frac{f^(n)(x)}{n!}h^{n-1}\;.\]\\
Auch die zweite Ableitung von $f$ an der Stelle $x$ möchten wir auf diese Weise nur mittels Funktionswerten von $f$ approximieren. Dies kann man jedoch nur mit der obigen Taylorentwicklung von $f$ an der Stelle $x_+$ nicht gut, da so die erste Ableitung zum Teil des Fehlers werden würde. Deshalb zieht man zusätzlich zur Taylorentwicklung an der Stelle $x_+$ die Taylorentwicklung von $f$ an der Stelle $x_-:=x-h\in[a,b]$
\[f(x_-)=\sum_{n=0}^{\infty}\frac{f^{(n)}(x)}{n!}(-h)^n=f(x)-f'(x)h+\frac{f''(x)}{2}h^2+...\]
in Betracht.\\
Durch Summation der beiden Taylorentwicklungen erhält man die Formel für die \textit{zweite finite Differenz}
\[(D_h^{(2)}f)(x)=\frac{f(x_+)-2f(x)+f(x_-)}{h^2}=f''(x)+\frac{f^{(4)}(x)h^2}{3*4}+...\;.\]

\subsection{Fehler und Schrittweite}
\label{ssec:schrittweite}
Die Fehler der numerischen Berechnung der Ableitungen gegenüber der exakten analytischen Berechnung setzen sich zusammen aus Verfahrensfehlern, d.h. Diskretisierungs- und Abbruchfehlern, und Rundungsfehlern.\\
Beim Verfahrensfehler hält es sich im Wesentlichen um den Unterschied zwischen dem exakten Wert der Ableitung an der Stelle x und dem exakten Wert des Differenzenquotienten an der Stelle x. Man ersetzt den Differentialquotienten durch den Sekantenanstieg.\\
- Computerarithmetik\\
Die Rundungsfehler sind darauf zurückzuführen, dass die Zahlendarstellung auf dem Computer nur mit endlicher Genauigkeit, näherungsweise an die eigentliche Zahl herankommt. \\

Ein Problem bei der Approximation mit Differenzenquotienten ist die Wahl der optimalen Schrittweite h. Ein zu großes h führt zu Verfahrensfehlern, ein zu kleines h zu Auslöschung.\\
Beim Vermindern der Schrittweite ist zu erwarten, dass der Fehler zunächst kleiner wird, da die Genauigkeit steigt. Da Maschinenzahlen jedoch nur eine endliche Genauigkeit bezitzen, muss man bei zu kleinen Schrittweiten mit Auslöschungen bei der Subtraktion rechnen und damit wieder mit einer sinkenden Genauigkeit. \\

\pagebreak \section{Experimente und Beobachtungen}
\label{sec:experimente}
- Untersuchen Konvergenzverhalten der Approximation der Ableitungen mit Finiten Differenzen\\

\pagebreak \section{Auswertung}
\label{sec:auswertung}
- Größenordunung des Fehlers bei 1. Abl größer, weil h, bei 2. kleiner, weil h hoch 2 für h<1\\
- bei zu wenig auswertungspunkten ungenauer plot, etc.\\

\pagebreak \section{Zusammenfassung}
\label{sec:zusammenfassung}
Dann die Zusammenfassung.

\pagebreak \section{Literatur}
\label{sec:literatur}
\end{document}