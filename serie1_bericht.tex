\documentclass{scrartcl}
\usepackage[utf8]{inputenc}
\usepackage[T1]{fontenc}
\usepackage[ngerman]{babel}
\usepackage{amsmath}
\usepackage{graphicx}

\begin{document}
% ----------------------------------------------------------------------
\title{PPI Bericht Serie 1}
\author{Marisa Breßler und Anne Jeschke}
\date{08.11.2019}
\maketitle
% ----------------------------------------------------------------------------
% Inhaltsverzeichnis:
\tableofcontents
% ----------------------------------------------------------------------------
% Gliederung und Text:
\section{Motivation}
\label{sec:motivation}
Hier kommt die Motivation.

\section{Theorie}
\label{sec:theorie}
- Approximation der Ableitung\\
- Taylorentwicklung\\
- 1. und 2. finite Differenz\\
- Fehlerabschätzungen

\section{Experimente}
\label{sec:experiment}
- Untersuchen Konvergenzverhalten der Approximation der Ableitungen mit Finiten Differenzen\\
- 1.: $g_1(x)=sin(x)/x$ auf $[\pi,3\pi]$ mit $p=1000$\\
- Vergleich der exakten und approximierten ersten und zweiten Ableitungen von g1 für $h = \pi/3$, $h = \pi/4$, $h = \pi/5$ und $h = \pi/10$\\
- Fehlerplots für verschiedene h, Vergleich erwartetes und actual Konvergenzverhalten\\

- 2.: $g_1(x)=sin(jx)/x$  Vergleich für kleinere j: \\
- bei j=0.5, 0.25 e2 fast konstant, e1 lin\\
- 0.1, 0.075 wie gewohnt\\
- 0.05, 0.01 wieder konstant\\
- 3.: Vergleich für größeres j:\\
- j=2, 4, 10, 20: e1 fast linear, e2 minimum bei 10-4, fehler insgesamt immer größer\\
- j=100-450: e2 minimum bei ca. 10-5, dann quadr , dann konstant?\\
- j=475: e1 konstant\\
- j=500 e2 fast gleich e1\\

- Amplitude der exakten 2. Abl immer viel größer als der Appr.
\section{Auswertung}
\label{sec:auswertung}
Dann die Auswertung.
\section{Zusammenfassung}
\label{sec:zusammenfassung}
Dann die Zusammenfassung.
\end{document}